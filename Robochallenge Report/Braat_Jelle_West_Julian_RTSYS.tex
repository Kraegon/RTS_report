%%%%%%%%%%%%%%%%%%%%%%%%%%%%%%%%%%%%%%%%%
% Structured General Purpose Assignment
% LaTeX Template
%
% This template has been downloaded from:
% http://www.latextemplates.com
%
% Original author:
% Ted Pavlic (http://www.tedpavlic.com)
%
% Note:
% The \lipsum[#] commands throughout this template generate dummy text
% to fill the template out. These commands should all be removed when 
% writing assignment content.
%
%%%%%%%%%%%%%%%%%%%%%%%%%%%%%%%%%%%%%%%%%

%----------------------------------------------------------------------------------------
%	PACKAGES AND OTHER DOCUMENT CONFIGURATIONS
%----------------------------------------------------------------------------------------
\documentclass[12pt]{article} % Default font size is 12pt, it can be changed here
\usepackage[dutch]{babel}
\newcommand{\sectionbreak}{\clearpage} % Starts every section on its own page
\usepackage[table,xcdraw]{xcolor}
\usepackage{geometry} % Required to change the page size to A4
\geometry{a4paper} % Set the page size to be A4 as opposed to the default US Letter
\usepackage{fancyhdr} % Required for custom headers
\usepackage{extramarks} % Required for headers and footers
\usepackage{lastpage} % Required to determine the last page for the footer
\usepackage{graphicx} % Required for including pictures

\usepackage{float} % Allows putting an [H] in \begin{figure} to specify the exact location of the figure
\usepackage{wrapfig} % Allows in-line images such as the example fish picture
\usepackage{lipsum} % Used for inserting dummy 'Lorem ipsum' text into the template

\linespread{1.2} % Line spacing

\graphicspath{{imgs/}}
%----------------------------------------------------------------------------------------
%	HEADER/FOOTER
%----------------------------------------------------------------------------------------


\pagestyle{fancy}
\lhead{Julian \textsc{West} \& Jelle Braat} % Top left header
\rhead{\textbf{[CONCEPT]} Robochallenge robot} % Top center header
\lfoot{\lastxmark} % Bottom left footer
\cfoot{} % Bottom center footer
\rfoot{Pagina\ \thepage\ van de\ \pageref{LastPage}} % Bottom right footer
\renewcommand\headrulewidth{0.4pt} % Size of the header rule
\renewcommand\footrulewidth{0.4pt} % Size of the footer rule

\setlength\parindent{0pt} % Removes all indentation from paragraphs

%----------------------------------------------------------------------------------------
%	TITLE PAGE
%----------------------------------------------------------------------------------------
\begin{document}

\begin{titlepage}
\pagenumbering{Roman}
\newcommand{\HRule}{\rule{\linewidth}{0.5mm}} % Defines a new command for the horizontal lines, change thickness here

\center % Center everything on the page
\includegraphics[scale=.1,keepaspectratio]{avans.pdf} \\
\textsc{\Large Avans Hogeschool Breda}\\[0.5cm] % Major heading such as course name
\textsc{\large Real-Time Systems (RTSYS)}\\[0.5cm] % Minor heading such as course title
\HRule \\[0.4cm]
{ \huge \bfseries \textbf{[CONCEPT]} Robochallenge Robot Design Document}\\[0.4cm] % Title of your document
\HRule \\[1.5cm]

\begin{minipage}{0.4\textwidth}
\begin{flushleft} \large
\emph{Auteurs:}\\
Julian \textsc{West} \\
Jelle \textsc{Braat} \\
\end{flushleft}
\end{minipage}
~
\begin{minipage}{0.4\textwidth}
\begin{flushright} \large
\emph{Leraren:} \\
Joli \textsc{van Kruijsdijk} \\ % Supervisor's Name
Hans \textsc{van der Linden} \\
%Paul \textsc{Lindelauf} %Indien Paul overneemt op 09/02
%Jan \textsc{Oostindie} %Onbekend of hij wel meehelpt
\end{flushright}
\end{minipage}\\[4cm]

{\large \today}\\[3cm] % Date, change the \today to a set date if you want to be precise
Versie: 0.1
\vfill % Fill the rest of the page with whitespace

\end{titlepage}

%----------------------------------------------------------------------------------------
% Preface
%----------------------------------------------------------------------------------------
\clearpage
\section*{Voorwoord}
\addcontentsline{toc}{section}{Voorwoord}
\lipsum[0-2]
%\\\\
\newpage
%----------------------------------------------------------------------------------------
%	TABLE OF CONTENTS
%----------------------------------------------------------------------------------------
%\setcounter{tocdepth}{1} % Uncomment this line if you don't want subsections listed in the ToC
\tableofcontents
\newpage
\pagenumbering{arabic}
\clearpage
%----------------------------------------------------------------------------------------
% Introduction
%----------------------------------------------------------------------------------------
\section{Introductie}
\label{sec:introduction}
\lipsum[0-3]
\newpage
%----------------------------------------------------------------------------------------
% Body
%----------------------------------------------------------------------------------------
\section{Eisen}
\label{sec:functionalities}
In dit hoofdstuk worden eisen bepaald aan het systeem van de Robochallenge robot. Deze eisen zijn onderverdeeld in de functionele, niet-functionele  en pseudo eisen. Onder functioneel wordt concrete handelingen verstaan, onder niet-functioneel kwaliteitseisen en onder pseudo eisen randvoorwaarden. 
\subsection{Functionele eisen}
De volgende eisen worden verder verdeelt in domeinen. Deze domeinen groeperen de handeling op basis van het doeleinde.
\subsubsection*{Domein actie}
\newcommand\litem[1]{\item{\bfseries #1\\}}
\begin{enumerate}
\litem{Bewegen 1} De robot moet autonoom bewegen, dus zonder enige ingrijpen van een persoon.
\litem{Bewegen 2} De robot moet kunnen bewegen in 8 assen van vrijheid (Vooruit, achteruit, links, rechts en diagonaal).
\litem{Grijpen 1} De robot moet een bal kunnen kunnen grijpen.
\litem{Grijpen 2} De robot moet een bal kunnen grijpen vanaf diverse invalshoeken.
\litem{Grijpen 3} De robot moet een bal kunnen grijpen op diverse hoogten.
\litem{Opslaan} De robot moet de ballen naar een intern reservoir kunnen brengen.
\newpage
\subsubsection*{Domein detectie}
\litem{Plaats 1} De robot moet zijn plaats in een ruimte kunnen detecteren.
\litem{Plaats 2} De robot moet de limieten (muren) van een ruimte kunnen bepalen.
\litem{Objecten} De robot moet de anderskleurige ballen vinden in relatie met zijn positie.
\litem{Kleur} De robot moet distinctie kunnen brengen in de kleuren van de ballen. 
\subsubsection*{Domein interactie}
\litem{Starten} De robot moet gestart worden door een enkele startknop.
\litem{Data verwerking} De robot mag vergaarde informatie draadloos versturen voor verwerking en weer ontvangen.
\litem{Noodknop} De robot moet voorzien zijn van een noodknop die onmiddellijk alle functionaliteiten van de robot stop legt.
\end{enumerate}
\newpage
\subsection{Niet-functionele eisen}
Deze lijst eisen volgt het zelfde formaat als de functionele eisen. Deze eisen zijn gegroepeerd op de functionele eisen waarop de niet-functionele eis wordt gesteld.
\subsubsection*{Domein grijpen}
\begin{enumerate}
\litem{Grip} De robot mag ballen die zijn gegrepen niet laten vallen.
\litem{Keuze} De robot moet de correcte bal pakken die de missie (beschreven in \textbf{TODO[ref]}) vereist.
\litem{Opslag 1} De robot moet een opslag hebben die voldoende groot is voor 17 ballen met een diameter van 7cm.
\litem{Opslag 2} De robot moet na het opslaan van een bal, deze niet kwijtraken.
\subsubsection*{Domein bewegen}
\litem{Stabiliteit} De robot mag niet omvallen.
\litem{Muren} De robot moet wanneer deze tegen een muur aanrijdt kiezen om een andere kant op te gaan.
\subsubsection*{Domein plaatsbepaling}
\litem{Nauwkeurigheid} De robot moet de plaats van een bal kunnen bepalen in 3 dimensies.
\subsubsection*{Domein kleur}
\litem{Precisie} De robot moet ballen kunnen onderscheiden tussen in hun primaire of secundaire kleur.

\end{enumerate}
\subsection{Pseudo eisen}
Ten slotte zijn er de pseudo eisen. Deze zijn niet gegroepeerd vanwege te weinig cohesie tussen deze eisen.
\begin{enumerate}
\litem{Grootte} De robot mag niet groter zijn dan 50x40cm.
\litem{Aandrijving} De robot mag alleen elektrisch aangedreven worden.
\litem{Stroomtoevoer} De robot moet voorzien zijn van zijn eigen stroom toevoer.
\litem{Spanning} De maximale interne spanning binnen de robot is 48 volt.
\litem{Licht} De robot mag geen verblindend licht gebruiken.
\litem{Misleiden} De robot mag niet voorzien zijn van onderdelen die de tegenstander kan misleiden.
\litem{Offensief} De robot mag niet voorzien zijn van offensieve middelen zoals rookbommen, stroomstoten of Elektromagnetische Pulse(EMP) wapens.
\litem{Noodknop} De noodknop op de robot moet zichtbaar en toegankelijk zijn. 
\end{enumerate}
\newpage
\%Mogelijk zouden er nog eisen kunnen worden gesteld aan procesmatige dingen, maar laten we die maar even buiten opzicht houden. Verder is er ook de test setup die beschreven kan worden, maar dat kan in zijn eigen hoofdstuk ofzo. Oh, en als aller laatste: die TODO daarboven wordt gevuld zodra we de missies in een later hoofdstuk hebben beschreven. dat kan met een ref{sec:mijnref} en label{sec:mijnref}. 

%----------------------------------------------------------------------------------------
% Conclusion?
%----------------------------------------------------------------------------------------
\section{Conclusie}
\label{sec:conclusion}
\lipsum[0-2]
\newpage
%----------------------------------------------------------------------------------------
%	Glossary
%----------------------------------------------------------------------------------------
%\clearpage
%\printglossaries
%----------------------------------------------------------------------------------------
%	Bibliography
%----------------------------------------------------------------------------------------
% \clearpage
% \bibliographystyle{plainurl}
% \nocite{*}
% \bibliography{Bibliography}
%----------------------------------------------------------------------------------------
%	Appendix
%----------------------------------------------------------------------------------------
%\addcontentsline{toc}{section}{Appendix A - Interns' hip assignment}
%\includepdf[pagecommand=\section*{Appendices}\subsection*{Appendix A - Internship assignment} The document including the assignment to the interns from Streamit,pages=-,scale=0.67]{appendices/Internship_assignments.pdf}
%----------------------------------------------------------------------------------------
%	End of document
%----------------------------------------------------------------------------------------
\end{document}

%figure example
%figures~\ref{fig:ganttFig1} and~\ref{fig:ganttFig2}.\\
%\begin{figure}
%\includegraphics{gantt4_2.png}
%\caption{Iterations of the project}
%\label{fig:ganttFig1}
%\end{figure}
%\begin{figure}
%\begin{center}
%%\includegraphics[angle=90,width=\textwidth,height=\textheight,keepaspectratio]{versie4.png}
%\includegraphics[angle=90,width=\textwidth,height=\textheight,keepaspectratio]{gantt4.png}
%\caption{Gantt chart of the planning}
%\label{fig:ganttFig2}
%\end{center}
%\end{figure}